\documentclass{article}
\usepackage{graphicx} % Required for inserting images
\usepackage{geometry}
\geometry{left=2.5cm,right=2.5cm,top=2.5cm,bottom=2.5cm}
\usepackage{amsfonts,amssymb,graphics,epsfig,verbatim,bm,latexsym,amsmath,url,amsbsy, amsthm}
\usepackage{enumitem}
\usepackage{listings}
\usepackage{xcolor}
\usepackage{array}
\usepackage{hyperref}
\usepackage{natbib}
\hypersetup{
    colorlinks=true,
    linkcolor=blue,
    filecolor=blue,
    urlcolor=blue,
}
% Define colors for the lines
\definecolor{linecolor}{rgb}{0,0,0}

% Custom environment for function name and short description
\newenvironment{cranfunction}[2]{%
    \textcolor{linecolor}{\rule{\linewidth}{0.4pt}}\par\vspace*{\fill}
    \texttt{#1} \hfill #2\par\nobreak\noindent
    \textcolor{linecolor}{\rule{\linewidth}{0.4pt}}\vspace*{\fill}\par
}{%

}
\title{\textbf{Package 'sreg'}}
\author{Juri Trifonov \\ \textit{Last version is available at} \href{https://github.com/yurytrifonov/sreg}{https://github.com/yurytrifonov/sreg}
}
\date{\today}
% Define a custom environment for function titles and descriptions
\newenvironment{funcdesc}[1]
    {\par\noindent\textbf{#1}\par\medskip\noindent\hrule\par\medskip\noindent}
    {\par\noindent\hrule\par\medskip}
    
\newenvironment{argdesc}{
    \par % start a new paragraph
    \noindent % no indentation
    \normalfont % typewriter font for the argument name
    
    \begin{tabular}{@{}p{0.3\textwidth}p{0.6\textwidth}@{}} % adjust the widths as needed
}{
    \end{tabular}
    \par % end the paragraph
   
}
\lstnewenvironment{rcode}[1][]
    {\lstset{language=R, basicstyle=\ttfamily, #1}}
    {}
\begin{document}
\maketitle

%\begin{center}
%\href{https://github.com/yurytrifonov/sreg}{https://github.com/yurytrifonov/sreg} 
%\end{center}
\subsection*{General Description}
The 'sreg' package for R offers a robust toolkit for estimating average treatment effects (ATEs) within the context of a stratified block randomization design under the covariate-adaptive randomization (CAR). Designed to accommodate scenarios with multiple treatments and cluster-level treatment assignments, the 'sreg' package not only provides ATE estimators but also includes features for calculating adjusted variance estimators developed in papers (Bugni, Canay, Shaikh; 2017), (Bugni, Canay, Shaikh, Meehan; 2023) and (Jiang, Linton, Tang, Zhang; 2023).

\subsection*{The function sreg()}

\begin{cranfunction}{sreg}{\textit{Estimates the ATE, standard errors, calculates p--values \& confidence intervals.}}
\end{cranfunction}
\subsubsection*{Description}
This function estimates the ATE(s) and the corresponding standard error(s) given the data provided. Multiple treatments, strata-based treatments, cluster-level treatments, and linear adjustments are supported. The function implements the appropriate estimator(s) given the data provided.
\subsubsection*{Usage}
\texttt{sreg(Y, S = NULL, D, G.id = NULL, Ng = NULL, X = NULL, Ng.cov = FALSE, HC1 = FALSE)}

\subsubsection*{Arguments}
\begin{argdesc}
    \texttt{Y} & a numeric $n \times 1$ vector of the \textit{observed outcomes}. \\
    \texttt{S} & a numeric $n \times 1$ vector of \textit{strata indicators}; if \texttt{NULL} then the estimator without strata is applied. \\
    \texttt{D} & a numeric $n \times 1$ vector of \textit{treatments}. \\
    \texttt{G.id} & a numeric $n \times 1$ vector of \textit{cluster indicators}; if \texttt{NULL} then the estimator without clusters is applied. \\
    \texttt{Ng} & a numeric $n \times 1$ vector of \textit{cluster sizes}; if \texttt{NULL} then the estimator without clusters is applied. \\
    \texttt{X} & a data frame with columns representing the covariate values for every observation; if \texttt{NULL} then the estimator without linear adjustments is applied. \\
    \texttt{Ng.cov} & a \texttt{TRUE/FALSE} logical argument indicating whether the \texttt{Ng} should be included as the only covariate in linear adjustments when \texttt{X = NULL}. \\
    \texttt{HC1} & a \texttt{TRUE/FALSE} logical argument indicating whether the small sample correction should be applied to the variance estimator. \\
\end{argdesc}

\subsubsection*{Value}
\noindent \textit{\textbf{Printed output}}

The function prints a "Stata-style" table containing the ATE estimates, corresponding standard errors, $t$-statistics, $p$-values, $95\%$ asymptotic confidence intervals, and significance indicators for different levels $\alpha$. The example of the printed output is provided below.


\begin{flushleft}
\begin{tabular}{lc}
\multicolumn{2}{l}{\texttt{Saturated Model Estimation Results under CAR}} \\
\multicolumn{2}{l}{\texttt{with clusters and linear adjustments}} \\
\texttt{Observations:} & \texttt{2910} \\
\texttt{Clusters:} & \texttt{100} \\
\texttt{Number of treatments:} & \texttt{2} \\
\texttt{Number of strata:} & \texttt{2} \\
\texttt{Covariates used in linear adjustments:} & \texttt{Ng, x\_1, x\_2} \\
\texttt{---} \\
\end{tabular}
\end{flushleft}

\begin{flushleft}
\begin{tabular}{lllllll}
\texttt{Coefficients:} \\
\texttt{Tau} & \texttt{As.se} & \texttt{T-stat} & \texttt{P-value} & \texttt{CI.left(95\%)} & \texttt{CI.right(95\%)} & \texttt{Significance} \\
\texttt{0.74881} & \texttt{0.13408} & \texttt{5.58497} & \texttt{0.00000} & \texttt{0.48602} & \texttt{1.01159} & \texttt{***} \\
\texttt{0.56782} & \texttt{0.14034} & \texttt{4.04610} & \texttt{0.00005} & \texttt{0.29276} & \texttt{0.84288} & \texttt{***} \\
\texttt{---} \\
\texttt{Signif. codes:} & \texttt{0 '***'} & \texttt{0.001 '**'} & \texttt{0.01'*'} & \texttt{0.05 '.'} & \texttt{0.1 ' '} & \texttt{1} \\
\end{tabular}
\end{flushleft}

\noindent\textit{\textbf{Return value}}

The function returns an object of class \texttt{sreg} that is a list containing the following elements:
\begin{itemize}
\item \texttt{tau.hat} -- a $1 \times |\mathcal A|$ vector of ATE estimates, where $|\mathcal A|$ represents the number of treatments.
\item \texttt{se.rob} -- a $1 \times |\mathcal A|$ vector of standard errors estimates, where $|\mathcal A|$ represents the number of treatments.
\item \texttt{t.stat} -- a $1 \times |\mathcal A|$ vector of $t$-statistics, where $|\mathcal A|$ represents the number of treatments.
\item \texttt{p.value} -- a $1 \times |\mathcal A|$ vector of corresponding $p$-values, where $|\mathcal A|$ represents the number of treatments.
\item \texttt{CI.left} -- a $1 \times |\mathcal A|$ vector of the left bounds of the $95\%$ as. confidence interval.
\item \texttt{CI.right} -- a $1 \times |\mathcal A|$ vector of the right bounds of the $95\%$ as. confidence interval.
\item \texttt{data} -- an original data of the form \texttt{data.frame(Y, S, D, G.id, Ng, X)}.
\item \texttt{lin.adj} -- a data frame representing the covariates that were used in implementing linear adjustments.
\end{itemize}

\subsubsection*{Example}
Here, we provide the empirical application example using the data from (Chong et al., 2016), who studied the effect of iron deficiency anemia on school-age children's educational attainment and cognitive ability in Peru. The example replicates the empirical illustration from (Bugni et al., 2019). For replication purposes, the data is included in the package and can be accessed by running \texttt{data("AEJapp")}.
\begin{rcode}
library(devtools)
install_github("yurytrifonov/sreg")
library(sreg)
library(dplyr)
library(haven)

# Data description
?AEJapp
# Upload the data from the package
data("AEJapp")
data <- AEJapp
# Prepare the data
Y <- data$gradesq34
D <- data$treatment
S <- data$class_level
data.clean <- data.frame(Y, D, S)
data.clean <- data.clean %>%
  mutate(D = ifelse(D == 3, 0, D))
# Look at the input data
head(data.clean)
     Y D S
1 11.2 1 1
2 12.4 0 3
3 11.9 0 5
4 13.1 0 1
5 13.4 2 2
6 10.7 0 1

# Look at the frequency table
table(D = data.clean$D, S = data.clean$S)
   S
D    1  2  3  4  5
  0 15 19 16 12 10
  1 16 19 15 10 10
  2 17 20 15 11 10
  
Y <- data.clean$Y
D <- data.clean$D
S <- data.clean$S

# Replicate the results from (Bugni et al, 2019)
result <- sreg::sreg(Y, S, D, HC1 = TRUE)

Saturated Model Estimation Results under CAR
Observations: 215 
Number of treatments: 2 
Number of strata: 5 
---
Coefficients:
       Tau   As.se   T-stat P-value CI.left(95%) CI.right(95%) Significance
1 -0.05113 0.20645 -0.24766 0.80440     -0.45577       0.35351             
2  0.40903 0.20651  1.98065 0.04763      0.00427       0.81379            *
---
Signif. codes:  0 ‘***’ 0.001 ‘**’ 0.01 ‘*’ 0.05 ‘.’ 0.1 ‘ ’ 1

## Besides that, it is possible to add linear adjustments (covariates)
x_1 <- data$pills_taken
x_2 <- data$age_months
data.clean <- data.frame(Y, D, S, x_1, x_2)
data.clean <- data.clean %>%
  mutate(D = ifelse(D == 3, 0, D))
  
  

# Look at the input data
head(data.clean)
     Y D S x_1      x_2
1 11.2 1 1   0 156.8460
2 12.4 0 3  16 186.4148
3 11.9 0 5   5 209.0513
4 13.1 0 1  21 146.2012
5 13.4 2 2   9 168.7721
6 10.7 0 1   5 146.0370
Y <- data.clean$Y
D <- data.clean$D
S <- data.clean$S
X <- data.frame(data.clean$x_1, data.clean$x_2)

# Results with linear adjustments
result <- sreg::sreg(Y, S, D, X, HC1 = TRUE)

Saturated Model Estimation Results under CAR
Observations: 215 
Number of treatments: 2 
Number of strata: 5 
---
Coefficients:
       Tau   As.se   T-stat P-value CI.left(95%) CI.right(95%) Significance
1 -0.02862 0.17964 -0.15929 0.87344     -0.38071       0.32348             
2  0.34609 0.18362  1.88477 0.05946     -0.01381       0.70598            .
---
Signif. codes:  0 ‘***’ 0.001 ‘**’ 0.01 ‘*’ 0.05 ‘.’ 0.1 ‘ ’ 1

\end{rcode}
\end{document}

